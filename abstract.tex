%%%%%%%%%%%%%%%%%%%%%%%%%%%%%%%%%%%%%%%%%%%%%%%%%%%%%%%%%%%%%%%%%%%%%%%%%%%%%

\chapter*{Abstract}
\addcontentsline{toc}{part}{Abstract}

Sergueï Mikhaïlovitch Liapounov (\foreignlanguage{russian}{Сергей Михайлович
Ляпунов}) est un compositeur, pianiste virtuose, chef d'orchestre, éthnomusicologue,
éditeur et pédagogue russe de la seconde moitié du XIX\ieme et du début du XX\ieme
siècle (1859-1924). Contrairement à Balakirev son mentor, il demeure très largement
oublié par l'histoire en dépit de sa contribution au développement de la musique
russe et en dépit de l'estime que lui portaient des artistes légendaires tels que
Ferruccio Busoni, Józef Hofmann, Vladimir Horowitz ou encore Ricardo Vi\~{n}es\dots{}
A l'heure d'aujourd'hui, il n'existe que très peu de sources écrites sur l'œuvre de
Liapounov et aucune d'entre-elles ne peut prétendre à l'exhaustivité.

Contemporain de Alexandre Glazounov, il fait parti des compositeurs qui restent
globalement insensibles à l'avant-garde musicale initiée par Scriabine ou Debussy.
Son œuvre s'inscrit dans l’esthétique russe inspirée par Glinka et ultérieurement
perpétuée par le \emph{Groupe des Cinq}\footnote{ou encore le \emph{puissant petit groupe}
constitué de Mili Balakirev, Alexandre Borodin, César Cui, Modeste Moussorgski
et Nikolaï Rimsky-Korsakov.}. Cependant, les références à Franz Liszt ou à Frédéric
Chopin sont extrêmement nombreuses (12 \emph{études d'exécution transcandante},
ballades, impromptus, préludes, nocturnes, mazurkas, valses, scherzos ou encore
barcarolles, etc...). Son catalogue comporte environ 80 partitions dont une
majorité écrites pour piano ou voix plus piano.\\

Dans la première partie de ce mémoire, nous essayons synthétiser la vie et
l'œuvre de Liapounov afin de positionner ce dernier face aux héritages du
romantisme européen et de la musique russe. Nous discuterons de quelques-uns
de ses principaux opus. Un intérêt particulier sera porté sur ses 12
\emph{études d'exécution transcendante}. Enfin, nous essayerons de dégager les
raisons de son anonymat.

La seconde partie de ce mémoire sera dédiée à l'étude de la sonate Op.27 qui,
par sa construction et ses dimensions, évoque la célèbre sonate en si mineur
de Franz Liszt.

%%%%%%%%%%%%%%%%%%%%%%%%%%%%%%%%%%%%%%%%%%%%%%%%%%%%%%%%%%%%%%%%%%%%%%%%%%%%%
