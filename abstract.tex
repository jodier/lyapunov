%%%%%%%%%%%%%%%%%%%%%%%%%%%%%%%%%%%%%%%%%%%%%%%%%%%%%%%%%%%%%%%%%%%%%%%%%%%%%

\chapter*{Abstract}
\addcontentsline{toc}{chapter}{Abstract}

Sergueï Mikhaïlovitch Liapounov (1859-1924) est un compositeur, pianiste virtuose,
chef d'orchestre, éthnomusicologue, éditeur et pédagogue russe. Contrairement à
Balakirev son mentor, il demeure très largement oublié par l'histoire en dépit
de l'estime que lui portaient des artistes légendaires tels que Josef Hofmann,
Ricardo Vi\~{n}es ou encore Vladimir Horowitz.
Contemporain de Tchaikovsky, il fait parti des compositeurs qui restent globalement
insenssiblent à l'avangarde musicale initiée par Scriabin ou Debussy. Son oeuvre
s'inscrit dans l'éstétique russe inspirée par Glinka et ultérieurement développée
par le Goupe des Cinq\footnote{ou encore \emph{le puissant petit groupe} constitué
de Mili Balakirev, Alexandre Borodin, César Cui, Modeste Moussorgski et Nikolaï
Rimsky-Korsakov}. Cependant, les références à Franz Liszt ou à Frédéric Chopin
sont extrèmement nombreuses (12 études d'exécution transcandantes, ballade,
mazurkas, polonaises, préludes, nocturnes, barcarolle, etc...). Son catalogue
comporte environ 80 oeuvres dont la moitié pour le piano.

Dans la première partie de ce mémoire, nous essayons syntétiser la vie et
l'oeuvre de Liapounov afin de positionner ce dernier face aux héritages du
romantisme européen et de la musique russe. Nous discuterons de quelques-uns
de ses principaux opus et enfin, nous essayerons de dégager les raisons de son
relatif anonymat.

La seconde partie de ce mémoire sera dédiée à l'étude de la sonate op. 27 qui,
par sa construction et ses dimenssions, évoque la célèbre sonate en si mineur
de Franz Liszt.

%%%%%%%%%%%%%%%%%%%%%%%%%%%%%%%%%%%%%%%%%%%%%%%%%%%%%%%%%%%%%%%%%%%%%%%%%%%%%
