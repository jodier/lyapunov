%%%%%%%%%%%%%%%%%%%%%%%%%%%%%%%%%%%%%%%%%%%%%%%%%%%%%%%%%%%%%%%%%%%%%%%%%%%%%

\chapter{Propos du mémoire}

Sergueï Mikhaïlovitch Liapounov (\foreignlanguage{russian}{Сергей Михайлович Ляпунов}) est un compositeur, pianiste virtuose, chef d'orchestre, éthnomusicologue, éditeur et pédagogue russe de la seconde moitié du XIX\ieme{} et du début du XX\ieme{} siècle (1859-1924). Contrairement à son mentor Balakirev, Liapounov demeure négligé par la musicologie en dépit de sa contribution au développement de la musique russe et de l'estime que lui portaient des artistes légendaires tels que Ferruccio Busoni, Józef Hofmann, Vladimir Horowitz ou Ricardo Vi\~{n}es\dots{} A l'heure d'aujourd'hui, il n'existe que peu de sources écrites sur l'œuvre de Liapounov et aucune d'elles ne prétend à l'exhaustivité.

Contemporain d'Alexandre Glazounov, il est resté en marge de l'avant-garde initiée par Scriabine ou Debussy. Son œuvre s'inscrit dans l’esthétique russe inspirée par Glinka puis perpétuée par le \emph{Groupe des Cinq}\footnote{ou encore le \emph{puissant petit groupe} constitué de Mili Balakirev, Alexandre Borodin, César Cui, Modeste Moussorgski et Nikolaï Rimski-Korsakov.}. Cependant, les références à Franz Liszt ou Frédéric Chopin sont extrêmement nombreuses. Ses œuvres portent les titres d'\emph{études d'exécution transcandante}, de ballades, d'impromptus, de préludes, de nocturnes, de mazurkas, de valses, de scherzos et autres barcarolles\dots{} Son catalogue comporte environ quatre-vingt partitions dont une majorité écrite pour piano ou voix et piano.\\

Dans la première partie de ce mémoire, nous synthétiserons la vie et l'œuvre de Liapounov afin de positionner le compositeur vis-à-vis des héritages du romantisme européen et de la musique russe. Nous discuterons de quelques-uns des principaux opus en portant un intérêt particulier aux \emph{études d'exécution transcendante}. De ce premier tableau se dégagera, peut-être, les raisons de son anonymat.

La seconde partie de ce mémoire sera dédiée à l'étude de la sonate en \emph{fa} mineur op.27 qui, par sa construction et ses dimensions, évoque la célèbre sonate en \emph{si} mineur de Franz Liszt.

%%%%%%%%%%%%%%%%%%%%%%%%%%%%%%%%%%%%%%%%%%%%%%%%%%%%%%%%%%%%%%%%%%%%%%%%%%%%%
