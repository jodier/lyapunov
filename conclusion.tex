%%%%%%%%%%%%%%%%%%%%%%%%%%%%%%%%%%%%%%%%%%%%%%%%%%%%%%%%%%%%%%%%%%%%%%%%%%%%%

\chapter{Conclusion}

Sergueï Mikhaïlovitch Liapounov (\foreignlanguage{russian}{Сергей Михайлович Ляпунов}) est issu d'une génération de compositeurs intermédiaire entre \emph{Groupe des Cinq} et Tchaïkovski d'une part, puis Scriabine et Stravinsky d'autre part. Disciple et ami fidèle de Balakirev, il n'a pas succombé aux temptations des courants contemporains. Dans son ouvrage bibliographique\cite{2}, Mikhaïl Shifman se réfère à Liapounov comme « \emph{le dernier des mohicans de l'école Balakirev} ».\\

Inspiré par la nature, les traditions folkloriques slaves et les grands romantiques parisiens (plus particulièrement Chopin et Liszt), Liapounov a significativement contribué au répertoire pianistique de son temps. En autres œuvres, citons ses deux symphonies op.11 et op.66, ses deux concertos pour piano op.4 et op.83, ses deux poèmes symphoniques op.36 et op.53, sa rhapsodie sur des thèmes ukrainiens op.28\dots{} Les \emph{études d'exécution transcendante} op.11 et la sonate op.27 se présentent comme un point de rencontre entre les influences européennes et les traditions russes. Ces deux opus constituent un sommet dans l'œuvre de Liapounov mais aussi dans la musique russe du début du XX\ieme{} siècle.\\

Oublié des interprètes et relativement négligée de musicologie, l'œuvre de Liapounov, colorée et authentique, mérite toute notre attention.

%%%%%%%%%%%%%%%%%%%%%%%%%%%%%%%%%%%%%%%%%%%%%%%%%%%%%%%%%%%%%%%%%%%%%%%%%%%%%
